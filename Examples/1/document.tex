\documentclass[12pt, a4paper]{article}
\usepackage{amsmath, amssymb}
\usepackage[dvips]{graphicx}
\usepackage{indentfirst}
\usepackage[utf8]{inputenc}
\usepackage[russian]{babel}
\usepackage{caption}
\DeclareCaptionLabelSeparator{dot}{. }
\captionsetup{justification=centering,labelsep=dot}
\setlength{\headheight}{0pt}
\setlength{\headsep}{0pt}\setlength{\oddsidemargin}{0mm}
\setlength{\evensidemargin}{10mm} \setlength{\topmargin}{-11mm}
\setlength{\textwidth}{\paperwidth} \addtolength{\textwidth}{-34mm}
\setlength{\textheight}{\paperheight}
\addtolength{\textheight}{-40mm}
\def\bs{\boldsymbol}
\usepackage{rotating, longtable, hhline}
\renewcommand{\baselinestretch}{1.34}
\newtheorem{remark}{Замечание}
\newtheorem{thm}{Теорема}
\newtheorem{lem}{Лемма}
\newtheorem{corollary}{Следствие}
\newcommand{\thb}{{\boldsymbol \theta}}
\newcommand{\eb}{{\bf e}}
\newcommand{\nut}{\nu_t, \: t \geqslant 0}
\newcommand{\pib}{{\boldsymbol \pi}}
\newcommand{\pb}{{\bf p}}
\newcommand{\intzi}{\int\limits_0^\infty}
\newenvironment{proof}{{\bf Доказательство.} \rm}{ $\Box$\par}
\newcommand{\T}{{\bf T}}
\newcommand{\sign}{{\rm sign }}
\newenvironment{definition}{{\bf Определение }\rm}






\begin{document}
\section{Исходные данные для отладки}

$N=3$.

Входной $MMAP$ определим следующим образом
Задаются  матрицы $D_0$ и $D$, которые  имеют вид
 $$
    D_0 = \left(\begin{array}{cc}
     -86 & \,0.01 \\
     0.02 & \,-2.76
    \end{array}\right),
   $$

    $$
    D = \left(\begin{array}{cc}
     85 & \,0.99 \\
     0.2 & \,2.54
\end{array}\right)
$$

Матрицу $D$ "растаскиваем"  на две матрицы $D_1 = 0.7 D$ и $D_2 =0.3 D.$

    $$
D_1 = \left(\begin{array}{cc}
59.5 & \,0.693 \\
0.14 & \,1.778
\end{array}\right)
$$

    $$
D_2 = \left(\begin{array}{cc}
25.5 & \,0.297 \\
0.06 & \,0.762
\end{array}\right)
$$

$\mu_1=20,\, \mu_2=10.$


\section{Результаты в ходе решения}

Формирование инфенитезимального генератора

Его блоки:
    $$
Q_{0,0} = \left(\begin{array}{cc}
     -86 & \,0.01 \\
0.02 & \,-2.76
\end{array}\right),
$$


$$
Q_{0,1} = \left(\begin{array}{cccc}
25.5 & \,0.297 & \,59.5 & \,0.693  \\
0.06 & \,0.762 & \,0.14 & \,1.178  \\
\end{array}\right),
$$

    $$
Q_{1,0} = \left(\begin{array}{cc}
10 & \,0 \\
0 & \,10 \\
20 & \,0 \\
0 & \,20
\end{array}\right),
$$

    $$
Q_{1,1} = \left(\begin{array}{cccc}
-11 & \,1 & \,0 & \,0  \\
0.22 & \,-10.22 & \,0 & \,0  \\
0 & \,0 & \,-21& \,1  \\
0 & \,0 & \,0.22 & \,-20.22  \\
\end{array}\right),
$$


 $$
G_0 = \left(\begin{array}{cc}
0.910873440285205& \,0.08912655971479502  \\
0.0196078431372549 & \,0.9803921568627452  \\
0.9528746465598492 & \,0.0471253534401508\\
0.010367577756833175 & \,0.9896324222431668 \\
\end{array}\right),
$$

Пересчитываем диагональные матрицы инфинетезимального генератора

    $$
Q_{0,0} = \left(\begin{array}{cc}
-6.0636775416190005 & \,6.063677541618994 \\
0.24142958665772882 & \,-0.24142958665772873 
\end{array}\right),
$$

Считаем матрицы $F_i$

    $$
F_0 = \left(\begin{array}{cc}
1 & \,0\\
0 & \,1
\end{array}\right),
$$



$$
F_1 = \left(\begin{array}{cccc}
2.323309625668449 & \,0.2563903743315509 & \,2.8351613100848256& \,0.17448868991517436   \\
0.0069593582887700536& \,0.07524064171122996 & \,0.007591800188501414 & \,0.0883081998114986   \\
\end{array}\right),
$$

Вычислим векторы $q_i$

$
q_0 = (0.004083672 \;\;\;0.6761417 ),
$

$
q_1 = (0.0095492014  \;\;\; 0.05950897 \;\;\; 0.012109048 \;\;\; 0.06355103),
$

$
q_2 = (0.011239560  \;\;\; 0.00418057492 \;\;\; 0.0273019902 \;\;\;  0.007959731   \;\;\; 0.01679074  \;\;\;  0.004042328 ),
$

$
q_3 = (0.0092572420   \;\;\;  0.000522203  \;\;\;   0.03332586   \;\;\;  0.0015594716  \;\;\;   0.04028129   \;\;\;  0.001621287  \;\;\;   0.01638924   \;\;\;  0.0005847308),
$\\



Распределение числа занятых приборов в системе
$$
q[0] = 0.6802254623191353\\
$$
$$
q[1] = 0.144718262443284\\
$$
$$
q[2] = 0.07151493175700846\\
$$
$$
q[3] = 0.10354134348057237\\
$$

$\lambda_1 = 12.426606557377042$

$\lambda_2 = 5.325688524590161$

\end{document}