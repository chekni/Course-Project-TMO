\documentclass[12pt,a4paper]{article}

\setlength{\hoffset}{-1in} \setlength{\voffset}{-1in}
\topmargin=25mm \headheight=0cm \headsep=0cm \oddsidemargin=30mm
\evensidemargin=30mm \textheight=23.7cm \textwidth=15.0cm \sloppy

\usepackage{amsbsy}
\usepackage{amsmath}
\usepackage{amssymb}
\usepackage{graphicx}
\usepackage{longtable,hhline}
\usepackage[russian]{babel}
\newcommand{\bs}{\boldsymbol}
\usepackage{xcolor}
\newcommand{\suml}{\sum\limits}

\begin{document}


\begin{center}
{\bf  СТАЦИОНАРНОЕ РАСПРЕДЕЛЕНИЕ СИСТЕМЫ МАССОВОГО ОБСЛУЖИВАНИЯ  $MAP/PH/1$}
\end{center}




\section{ Математическая модель}


Рассматривается  однолинейная система массового обслуживания с бесконечным буфером.
Заявки поступают в систему в  $MAP$ (Markovian Arrival Process)-потоке, который является ординарным аналогом  $BMAP$ 
(Batch Markovian Arrival Process).
 Поступление заявок в этом потоке возможно только в моменты скачков некоторой
неприводимой цепи Маркова $\nu_t,\,t\geq 0,$ с непрерывным временем
и конечным пространством состояний $\{0,1,\dots,W\}$, которая
называется управляющим процессом $MAP$-потока. Интенсивности
переходов процесса $\nu_t,\,t\geq 0$, сопровождающиеся поступлением
заявки  (не сопровождающиеся поступлением
заявок), задаются элементами матрицы $D_1$ (недиагональными
элементами матрицы $D_0$).
 При этом матрица
$D_0+D_1$ является неприводимым инфинитезимальным генератором цепи
Маркова $\nu_t,\,t\geq 0$. Вектор-строка $\bs \theta$ стационарного
распределения этой цепи является решением системы уравнений
$\bs\theta (D_0+D_1)= {\bf 0},\, \bs\theta{\bf e }= 1$, где $\bf e$  --
вектор-столбец, состоящий из единиц, $\bf 0$-вектор-строка, состоящая из нулей. Интенсивность  $\lambda$
поступления заявок в стационарном $MAP$-потоке имеет вид $\lambda =
\bs\theta D_1 {\bf e}$. Коэффициент вариации $c_{var}$
длины интервала  между моментами поступления последовательных групп
определяется формулой $c_{var}^2 = 2\lambda
\bs\theta(-D_0)^{-1}{\bf e} - 1$. Коэффициент корреляции $c_{cor}$
длин двух соседних интервалов вычисляется следующим образом:
$c_{cor} = (\lambda\bs\theta(-D_0)^{-1}D_1(-D_0)^{-1}{\bf e}
- 1)/c_{var}^2$.
Более подробное описание $BMAP$-потока можно найти в
\cite{bib:Luc91}.

Время обслуживания заявки на  приборе  имеет $PH$ (Phase type) распределение
  с неприводимым представлением
$({\boldsymbol \beta},S)$  и управляющим процессом $m_t,
\,t \ge0,$ с пространством состояний $\{1,\dots,M, M+1\}$, где
состояние $M+1$ является поглощающим.
 Это означает следующее.
 Время обслуживания  интерпретируется как время, за которое цепь Маркова
 $m_t$, $t \geq 0,$  достигнет  поглощающего состояния $M+1$. Переходы цепи $m_t$, $t \geq 0,$ в пространстве состояний
 $\{1,\ldots,M\}$ задаются субгенератором $S$, а интенсивности переходов в поглощающее состояние задаются вектором  $\bs 
 S_0 = -S {\bf e}$. Когда обслуживание начинается, состояние процесса  $m_t$, $t \geq 0,$ выбирается  в пространстве 
 состояний $\{1,\ldots,M\}$ на основании вероятностного вектора-строки $\bs{\beta}$. Полагаем, что  матрица $S+\bs
 S_0\bs{\beta} $ неприводима. Интенсивность обслуживания задается как
 $\mu=-(\bs{\beta}S^{-1}{\bf e})^{-1},$ среднее время обслуживания $b_1=\mu^{-1}.$

Дисциплина выбора из очереди на обслуживание FIFO.

{\bf Цель работы:} вычислить стационарное распределение вероятностей состояний системы (векторы ${\bf p}_i, i\geq0$) и 
основные характеристики производительности. Построить графики зависимостей характеристик от параметров системы.



\section{ Цепь Маркова, описывающая функционирование системы}



Пусть в момент $t$

$\bullet$ $i_t$ - число заявок  в системе, $i_t\geq 0$,



$\bullet$ $m_t$ - состояние управляющего процесса РН - обслуживания на   занятом приборе, $ m_t=\overline{1,M};$


$\bullet$ $\nu_t$  - состояние управляющего  процесса  входящего $MAP$ потока,
$\nu_t=\overline{0,W}.$


Процесс функционирования системы описывается неприводимой цепью Маркова  $\xi_t$ с пространством состояний

\begin{equation*}
X=\{(i, \nu),\,  i=0,  \nu=\overline{0,W}\}\bigcup
    \end{equation*}
    \begin{equation*}
\{(i, \nu, m),\,  i\geq1, \nu=\overline{0,W}, m=\overline{1,M} \}.
    \end{equation*}




    Далее будем предполагать, что состояния цепи $\xi_t, t\geq 0,$  упорядочены в лексикографическом порядке.  Обозначим 
    через  $Q_{i,j}$ матрицу интенсивностей переходов цепи из состояний, соответствующих значению  $i$ первой (счетной) 
    компоненты в состояния, соответствующие значению  $j$ этой компоненты, $i, j\geq 0.$




{\bf Лемма  1.}
Рассматриваемая цепь Маркова $\xi_t,\, t\ge 0$, является векторным
процессом гибели и размножения ($QBD$). Инфинитезимальный генератор  $Q$ этой цепи  имеет  блочную трехдиагональную 
структуру
$$
Q= \left ( \begin{array}{ccccc}
Q_{0,0} &Q_{0,1} & O&O& \ldots   \\
Q_{1,0} &Q_1 & Q_2&O& \ldots \\
O& Q_0 &Q_1 &Q_2& \ldots  \\
O&O& Q_0 &Q_1 &\ldots  \\
\vdots &\vdots &\vdots &\vdots  &\ddots  \\
\end{array}\right),
$$
   где ненулевые блоки имеют следующий вид:
 $$
Q_{0,0} =D_0,\quad Q_{0,1} =D_1,
$$
$$
Q_{1,0}=I_{\bar{W}}\otimes {\bs S}_0,
$$
$$
Q_0 =I_{\bar{W}}\otimes {\bs S}_0{\bs \beta},\qquad
   Q_1=D_0\oplus S,
 \qquad
       Q_2 = D_1\otimes I_M,
$$
где $\otimes,\, \oplus$ - символы кронекерова произведения и суммы матриц соответственно (см. \cite{graham}), 
$\,\,\bar{W}=W+1.$



{\bf Следствие  1.}
  Цепь Маркова  $\xi_t, t \geq 0,$ принадлежит классу квазитеплицевых цепей с непрерывным временем (КТЦМ), см.  
  \cite{kd}.


Доказательство  следует из вида генератора, заданного леммой  1,  и определения КТЦМ, данного в \cite{kd}.



\section{Условие эргодичности }


{\bf Теорема  1.}
  Цепь Маркова $\xi_t$ эргодична тогда и только тогда, когда выполняется неравенство
$$
\rho=\frac{\lambda}{\mu}<1.
$$
Здесь
$$
\lambda =\bs\theta D_1 {\bf e},
$$
где $\bs\theta$--вектор-строка порядка $\bar{W}$, который находится как единственное решение СЛАУ
$$
\bs\theta (D_0+D_1)= {\bf 0},\, \bs\theta{\bf e} = 1,
$$

$$
\mu=-(\bs{\beta}S^{-1}{\bf e})^{-1}.
$$







\section{ Стационарное распределение }

Предполагаем, что условие эргодичности, заданное теоремой 1 выполняется, что гарантирует, что в рассматриваемой системе 
не накапливается бесконечной очереди и существуют
стационарные вероятности состояний системы, задаваемые как

\begin{equation*}
 p (0,\nu)=\lim\limits_{t \to \infty} P\{i_t=0,\nu_t=\nu\},\,n=0,1,\
 \nu=\overline{0,W};
    \end{equation*}



 $$
  p (i,  \nu, m)=\lim\limits_{t \to \infty} P\{i_t=i,\, \nu_t=\nu, m_t=m\},\,
    i\geq1,\, n=0,1, \nu=\overline{0,W}, \,m=\overline{1,M}.
    $$

 Упорядочим вероятности в  лексикографическом порядке компонент и сформируем векторы этих вероятностей
${\bf p}_i, \,i\geq0.$

Порядки этих векторов равны:

${\bf p}_i\sim\,\, \bar{W}(1+M),\,i\geq1.$



Векторы стационарных вероятностей $\mathbf{p}_i, i\geq0,$
вычисляются по следующему алгоритму (cм. \cite{kd}).


{\it АЛГОРИТМ}

\begin{itemize}
    \item[1)] Находим матрицу $G$ как единственное минимальное неотрицательное
    решение уравнения
    $$
    Q_0+Q_1G+Q_2G^2=O.
    $$

    {\bf Замечание 1}. Это матричное уравнение можно решать методом итераций
 $$
 G^{(n+1)}=(-Q_1)^{-1}[Q_0+Q_2 (G^{(n)})^2],
 $$
 где $G^{(0)}=I_{\bar{W}(1+M)}.$ Останавливаемся, когда становится  $\| G^{(n+1)}-G^{(n)})\|<\varepsilon_G.$

    \item[2)] Находим матрицу $G_0$ из уравнения обратной рекурсии:
    $$
    G_0=-\left(
    Q_1+Q_2G\right)^{-1}Q_{1,0}.
    $$

     \item[3)] Вычисляем матрицы $\bar{Q}_{i,j}$  по формулам

     \begin{align*}
              \bar Q_{i,i} =     \left\{
                \begin{array}{l}
     Q_{0,0} + Q_{01}G_0,\,\,i=0,\\
    Q_1 + Q_2G,\,i\geq 1,\\
                \end{array}
            \right.
    \end{align*}
    \begin{align*}
              \bar Q_{i,i+1} =     \left\{
                \begin{array}{l}
     Q_{0,1},\,i=0,\\
    Q_2,\,i\geq 1.\\
                \end{array}
            \right.
    \end{align*}



    \item[4)] Находим матрицы $F_{i}$  из рекуррентных соотношений:
    $$
    F_{0}=I,F_{i}=F_{i-1}\bar{Q}_{i-1,i}\left( -\bar{Q}_{i,i}
    \right)^{-1},i\ge 1.
    $$

    \item[5)] Вычисляем вектор $\bs{p}_{0}$ как единственное решение
    СЛАУ:
    $$
     \bs{p}_{0}(-\bar{Q}_{0,0})=0,\,\,
        \bs{p}_{0}\displaystyle\sum_{i=0}^{\infty} F_{i}{\bf e}=1.
         $$
    \item[6)] Вычисляем векторы $\bs{p}_{i}$  по формулам
   $
    \bs{p}_{i}=\bs{p}_{0}F_{i},i\geq 0.
   $



\end{itemize}

{\bf Замечание.} При выполнении шагов 3-6 как-то нужно выбрать значение $i$, при котором мы заканчиваем счет. Чтобы это 
сделать,  мы должны понимать, что при возрастании $i$ норма матриц $F_i$ убывает. Нам надо, чтобы эта норма была очень 
малой. Поэтому в качестве предельного значения $i,$ при котором мы заканчиваем  счет, берем такое, при котором уже будет 
выполняться неравенство $\|F_i\|< \epsilon_F.$


Преимуществом этого алгоритма является отсутствие операции вычитания в рекурсиях. Все матрицы, фигурирующие в алгоритме, 
являются неотрицательными.
Это обеспечивает устойчивость вычислений при компьютерной реализации алгоритма.

\section{Характеристики производительности системы }

 Вычислив векторы стационарных вероятностей ${\bf p}_i, \, i \geq 1,$ можно вычислить  также различные характеристики 
 производительности системы.



\begin{itemize}


\item[$\bullet$] Вероятность того, что в системе  находится $i$ заявок
$\quad
p_i={\bf p}_i{\bf e}, i\geq0.
$

 \item[$\bullet$] Среднее число заявок в системе
 $\quad
L= \sum\limits_{i=1}^\infty i p_i.
 $


\item[$\bullet$] Вероятность того, что поступившая заявка застанет $i$ заявок  в системе
$$
p_i^{(arrival)}=\frac{1}{\lambda}{\bf p}_iD_1{\bf e}.
   $$

 \item[$\bullet$]  Среднее время пребывания

 $$
 \bar{w}=-\sum\limits_{i=1}^\infty  {\bf p}_i({\bf e}_{\bar{W}}\otimes I_M)S^{-1} {\bf e}_M+Lb_1.
 $$


  \end{itemize}

\section{Численные примеры }

Исходные данные должны вводится с экрана. К ним относятся:

1. $W$;

2. Матрицы $D_0,\, D_1$;

3. Число $M$;

4. Вектор-строка  $\bs \beta, $ матрица $S$.

5. Точность $\varepsilon_G.$


\vspace{1cm}

Для отладки взять

1. $W=1$;

2. $ D_0 = \left(\begin{array}{cc}
     -8 & 1 \\
     1 & -11
    \end{array}\right),
    D_1= \left(\begin{array}{cc}
     2 & 5 \\
     4 & 6
    \end{array}\right).$

3.  $M=2$;

4. $\bs \beta=(1,0), $ $S= \left(\begin{array}{cc}
     -30 & 30 \\
     0 & -30
    \end{array}\right)$.

5. $\varepsilon_G=10^{-8}.$



\begin{thebibliography}{99}

\bibitem{bib:Luc91}
  Lucantoni~ D.M.~(1991).  New results on the single server
queue with a batch Markovian arrival process.  {\sl Communications in
Statistics-Stochastic Models}. Vol.~{\bf 7}, pp.~ 1-46.


\bibitem{graham}
 Graham A.  Kronecker Products and Matrix Calculus with
 Applications.  Cichester:
Ellis  Horwood, 1981.

\bibitem{kd}
 Klimenok~V.I.,   Dudin~A.N.~(2006).  Multi-dimensional asymptotically quasi-Toeplitz
Markov chains and their application in queueing theory. {\sl  Queueing Systems}. Vol.~{\bf 54}, pp.~ 245-259.


%\bibitem{neuts}
%  Neuts~M.~ (1981). {\sl  Matrix-geometric Solutions in Stochastic Models -- An
%Algorithmic Approach}.  Johns Hopkins University Press, Baltimore.


 \end{thebibliography}







%%%%%%%%%%%%%%%%%%%%%%%%%%%%%%%%%%%%%%%%%%%%%%%%%%%%%%%%%%%%%%%%%%%%%%%%%%%%%%%%%%%%%%%%%
\end{document}