\documentclass[12pt, a4paper]{article}
\usepackage{amsmath, amssymb}
\usepackage[dvips]{graphicx}
\usepackage{indentfirst}
\usepackage[cp1251]{inputenc}
\usepackage[russian]{babel}
\usepackage{caption}
\DeclareCaptionLabelSeparator{dot}{. }
\captionsetup{justification=centering,labelsep=dot}
\setlength{\headheight}{0pt}
\setlength{\headsep}{0pt}\setlength{\oddsidemargin}{0mm}
\setlength{\evensidemargin}{10mm} \setlength{\topmargin}{-11mm}
\setlength{\textwidth}{\paperwidth} \addtolength{\textwidth}{-34mm}
\setlength{\textheight}{\paperheight}
\addtolength{\textheight}{-40mm}
\def\bs{\boldsymbol}
\usepackage{rotating, longtable, hhline}
\renewcommand{\baselinestretch}{1.34}
\newtheorem{remark}{Замечание}
\newtheorem{thm}{Теорема}
\newtheorem{lem}{Лемма}
\newtheorem{corollary}{Следствие}
\newcommand{\thb}{{\boldsymbol \theta}}
\newcommand{\eb}{{\bf e}}
\newcommand{\nut}{\nu_t, \: t \geqslant 0}
\newcommand{\pib}{{\boldsymbol \pi}}
\newcommand{\pb}{{\bf p}}
\newcommand{\intzi}{\int\limits_0^\infty}
\newenvironment{proof}{{\bf Доказательство.} \rm}{ $\Box$\par}
\newcommand{\T}{{\bf T}}
\newcommand{\sign}{{\rm sign }}
\newenvironment{definition}{{\bf Определение }\rm}






\begin{document}
	
	\begin{center}
		{\bf Система массового обслуживания $MMAP/M_1,M_2/N/0$}
	\end{center}
	
	
	\section{ Описание системы}
	
	Рассматривается $N$-линейная система массового обслуживания без буфера.
	Запросы разных типов поступают в $MMAP$ -потоке под управлением
	неприводимой цепи Маркова с непрерывным временем
	$\nu_{t}$, $t\ge0$, которая принимает значения в множестве
	$\{0,1,2,\ldots, W\}$. Цепь
	$\nu_{t}$ пребывает в состоянии $\nu$ в течение экспоненциально
	распределенного времени с параметром $\lambda_{\nu}$,
	$\nu= \overline{0, W}$, после чего с вероятностью $p_{k}(\nu,\nu')$
	переходит в состояние $\nu'$ и генерируется заявка $k$-го типа,
	$k\in\{1,2,\ldots,K\}$, или, с вероятностью $p_{0}(\nu,\nu')$, цепь
	переходит в состояние $\nu'$ без генерации заявки, причем
	$p_{0}(\nu,\nu)=0$. Процесс $\nu_{t}$ - называется
	управляющим процессом $MMAP$-потока.
	Для указанных  вероятностей выполняются естественные ограничения:
	$
	\sum\limits_{k=1}^{K}\sum\limits_{\nu'=0}^{ W}p_{k}(\nu,\nu')=1,
	\;  \nu,\nu'=\overline{0,W}.
	$
	Таким образом, $MMAP$-поток задается:
	\begin{itemize}
		\item[-] размерностью пространства состояний управляющего
		процесса, $ W+1$;
		\item[-] количеством типов заявок, $K$;
		\item[-] интенсивностями времен пребывания управляющего процесса
		в соответствующих состояниях, $\lambda_{\nu}$,
		$\nu= \overline{0, W}$;
		\item[-] вероятностями переходов, $p_{k}(\nu,\nu')$,
		$k= \overline{1,K}$, $\nu,\nu'= \overline{0, W}$.
	\end{itemize}
	
	По аналогии с $BMAP$-потоком всю информацию и $MMAP$-е удобно
	хранить в виде набора матриц $D_{k}$, $k= \overline{1,K},$ порядка
	$ (W+1)\times (W+1),$ элементы которых определяются как
	$$
	(D_{k})_{\nu,\nu'}=\begin{array}{ll}\lambda_{\nu}p_{k}(\nu,\nu'),&\nu,\nu'= \overline{0, W},k= 
	\overline{1,K},\end{array}
	$$
	$$
	(D_{0})_{\nu,\nu'}=\left\{\begin{array}{ll}
	\lambda_{\nu}p_{0}(\nu,\nu'),&\nu\ne\nu',\nu,\nu'= \overline{0,  W},\\
	-\lambda_{\nu},&\nu=\nu'= \overline{0,  W}.
	\end{array}\right.
	$$
	
	Нетрудно видеть, что элементами матриц $D_{k}$, $k>0,$ являются
	интенсивности переходов процесса $\nu_{t}$, сопровождающиеся
	генерацией заявки $k$-го типа. Аналогичный смысл имеют
	недиагональные элементы матрицы $D_{0}$, а диагональные элементы
	этой матрицы - взятые с противоположным знаком интенсивности выхода
	процесса $\nu_{t}$ из соответствующих состояний.
	
	Естественное требование к матрицам $D_k$, $k\ge1,$ состоит в том, что
	не все они нулевые. При выполнении этого требования матрица $D_0$
	является невырожденной и, более того, устойчивой, т.к. все ее
	собственные значения имеют отрицательную действительную часть.
	
	Матрицы $D_k$, $k\ge0,$ можно задавать их матричной производящей
	функцией:
	$$
	D(z)=\sum\limits_{k=0}^{K}D_{k}z^k, \; |z|<1.
	$$
	
	Отметим, что значение этой функции в точке $z=1$ --- матрица $D(1)$
	--- является инфинитезимальным генератором управляющего процесса
	$\nu_t$, $t\ge0$. Стационарное распределение этого процесса,
	представленное в виде вектор-строки $\bs{\theta}$, определяется как
	решение системы линейных алгебраических уравнений:
	$$
	\left\{\begin{array}{l}\bs{\theta}D(1)=\bs{0},\\
	\bs{\theta}{\bf e}=1.\end{array}\right.
	$$
	
	Интенсивность $\lambda_{k}$ поступления запросов $k$-го типа
	в $MMAP$-потоке задается формулой:
	$$
	\lambda_{k}=\bs{\theta}D_{k}{\bf e},
	$$
	а суммарная  интенсивность поступления запросов, $\lambda$,
	по формуле:
	$$
	\lambda=-\bs{\theta}\sum\limits_{k=1}^K D_{k}\bs{e}.
	$$
	
	
	
	Дисперсия $v_k$ длин интервалов между моментами  поступления групп
	запросов $k$-го типа вычисляется по формуле
	$$
	v^{(k)} = \frac{2\bs{\theta}(-D_0-
		D_k)^{-1}{\bf e}}{\lambda_k} -
	\biggl(\frac{1}{\lambda_k}\biggr)^2,\; k=\overline{1,K}.
	$$
	
	
	Коэффициент корреляции  $c_{cor}^{(k)}$ длин двух соседних
	интервалов между моментами  поступления групп запросов $k$-го типа
	вычисляется по формуле
	$$
	c_{cor}^{(k)}=\biggl[\frac{\bs{\theta}(-D_0-
		D_k)^{-1}{\bf e}}{\lambda_k} D_k(-D_0-D_k)^{-1}{\bf
		e}-\biggl(\frac{1}{\lambda_k}\biggr)^2\biggr]v_k^{-1},\;
	k=\overline{1,K}.
	$$
	Более подробное описание $MMAP$  можно найти, например, в \cite{he}.
	
	Отметим, что стационарный пуассоновский поток является частным случаем
	$MMAP$-потока при $ W=0$, $K=1$, $D_0=-\lambda$, $D_1=\lambda$.
	
	В данном отчете мы предполагаем, что в систему поступают в $MMAP$-потоке запросы двух типов, т.е. $K=2$.
	
	Время обслуживания запроса $k$-го типа   имеет экспоненциальное распределение
	с параметром $\mu^{(k)},  k=1,2.$
	
	Если запрос, поступающий  в систему,  застает  все приборы занятыми, то он покидает тандем
	навсегда.
	
	Нашей целью является
	расчет стационарного распределения системы, а также
	вычисление  вероятностей потерь запросов разных типов.
	
	
	
	
	
	
	
	\subsection{ Стационарное распределение системы  $MMAP/M/N/N$}
	
	
	
	Функционирование системы  $MMAP/M/N/N$ с двумя типами запросов описывается цепью Маркова
	$\eta_t= \{n_t, l_t, \nu_t\}$ где $n_t$ -- число занятых приборов,  $l_t$ -- число  приборов, занятых обслуживанием 
	запросов 1 типа, $\nu_t,\,\nu_t \in \{0,\dots,W\}$   -- состояние управляющего
	процесса $MAP$ в  момент времени $t.$
	
	Перенумеруем состояния цепи  в лексикографическом порядке. Тогда
	инфинитезимальный  генератор этой цепи определяется как
	$$
	Q= \left(\begin{array}{cccccc}
	Q_{0,0} & Q_{0,1} & O & \ldots & O & O\\
	Q_{1,0}& Q_{1,1} & Q_{1,2} &  \ldots & O & O\\
	O & Q_{2,1} & Q_{2,2}&  \ldots & O & O\\
	\vdots & \vdots & \vdots &  \ddots & \vdots & \vdots\\
	O & O & O & \ldots &Q_{N-1, N-1} &Q_{N-1, N} \\
	O & O & O & \ldots & Q_{N, N-1} & Q_{N, N}\end{array}\right).
	$$
	
	где
	
	$$
	Q_{0,0}=D_0, Q_{n,n}=I_{n+1}\otimes D_0-diag\{(n-l)\mu^{(2)}+l\mu^{(1)},\, l=\overline{0, n}\}\otimes  I_{W+1}, 
	n=\overline{0, N-1},
	$$
	$$
	Q_{N, N}=I_{N+1}\otimes D(1)-diag\{(N-l)\mu^{(2)}+l\mu^{(1)},\, l=\overline{0, N}\}\otimes  I_{W+1},
	$$
	$$
	Q_{n, n-1}=\left(\begin{matrix}
	diag\{(n-m)\mu^{(2)},\, m=\overline{0, n-1}\}\\
	{\bf 0}_n
	\end{matrix}\right)
	+\left(\begin{matrix}
	{\bf 0}_n\\
	diag\{m\mu^{(1)},\, m=\overline{1, n}\},
	\end{matrix}\right), \, n=\overline{1, N},
	$$
	
	$Q_{n, n+1}$ -- матрица порядка $(W+1)(n+1)\times (W+1)(n+2)$ вида
	$$
	Q_{n, n+1}=\left(\begin{matrix}
	D_2&D_1&O& \ldots &O&O&\\
	O&D_2&D_1& \ldots &O&O&\\
	\vdots & \vdots & \vdots &\ddots & \vdots & \vdots \\
	O&O&O& \ldots &D_2&D_1\\
	\end{matrix}\right), n=\overline{0, N-1}.
	$$
	
	
	Пусть ${\bf p}$ является вектором-строкой стационарного
	распределения  вероятностей состояний цепи. Этот вектор определяется
	как единственное  решение
	системы линейных алгебраических уравнений
	$$
	{\bf p}Q={\bf 0},\,\,{\bf p}{\bf e}=1.
	$$
	
	В случае большой размерности данной системы для ее решения
	целесообразно использовать специальный  алгоритм, предложенный в \cite{KlimKimOrlDud} и основанный на идее сенсорных 
	цепей Маркова.
	
	Представим вектор ${\bf p}$  как
	${\bf p}=({\bf p}_0, {\bf p}_1,\ldots, {\bf p}_N),$ где
	вектор ${\bf p}_i$ имеет порядок  $(i+1)(W+1),$   $ i=0,\ldots,N.$
	
	
	{\it АЛГОРИТМ}
	
	\begin{itemize}
		
		\item[1)] Находим матрицы $G_{N-1},G_{N-2},\ldots, G_0,$ из уравнения обратной рекурсии:
		$$
		G_{i}=\left(
		-Q_{i+1,i+1}-Q_{i+1,i+2}G_{i+1}\right)^{-1}Q_{i+1,i},i=N-2, N-1,\ldots,0,
		$$
		где полагаем
		$$
		G_{N-1}=(-Q_{N,N})^{-1}Q_{N,N-1}.
		$$
		
		\item[3)] Вычисляем матрицы $\bar{Q}_{i,i}, \bar{Q}_{i,i+1}$ по формулам
		
		$$
		\overline{Q}_{i,i}=Q_{i,i}+Q_{i,i+1}G_i,
		$$
		$$
		\overline{Q}_{i,i+1}=Q_{i,i+1},i=\overline{0,N-1}.
		$$
		
		\item[4)] Находим матрицы $F_{i}$  из рекуррентных соотношений:
		$$
		F_{0}=I,F_{i}=F_{i-1}\bar{Q}_{i-1,i}\left( -\bar{Q}_{i,i}
		\right)^{-1},\,i=\overline{1,N}.
		$$
		
		\item[5)] Вычисляем вектор ${\bf p}_{0}$ как единственное решение
		СЛАУ:
		$$
		{\bf p}_{0}(-\bar{Q}_{0,0})=0,\,\,
		{\bf p}_{0}\displaystyle\sum_{i=0}^{\infty} F_{i}{\bf e}=1.
		$$
		\item[6)] Вычисляем векторы ${\bf p}_i$  по формулам
		$
		{\bf p}_i={\bf p}_0 F_{i},i\geq 0.
		$
		
	\end{itemize}
	
	
	Обратим внимание, что операции вычитания не присутствуют  в данном
	алгоритме, а все обратные  матрицы существуют и неотрицательны.
	Таким образом, алгоритм является  численно устойчивым.
	
	\section{ Характеристики производительности системы}
	
	Рассчитав стационарное  распределение системы, можно
	найти  ряд важных стационарных характеристик производительности
	системы. Приведем некоторые из них.
	
	\begin{itemize}
		\item[$\bullet$] Распределение числа занятых приборов в системе
		
		$$
		p_i={\bf p}_i{\bf e}, \,i=\overline{0,N}.
		$$
		
		\item[$\bullet$] Векторы ${\bf p}_i^-$ порядка $W+1,$ $j$-я компонента есть вероятность того, что $i$ приборов 
		заняты и $MMAP$ находится в состоянии $j-1$
		
		$$
		{\bf p}_i^-={\bf p}_i ({\bf e}_{i+1}\otimes I_{\bar{W}}),\,i=\overline{0,N}.
		$$
		
		
		
		
		
		
		\item[$\bullet$] Среднее  число занятых приборов
		
		$$
		N_{busy}=\sum\limits_{i=1}^N i p_i.
		$$
		
		\item[$\bullet$] Вектор ${\bf q}_l$ порядка $W+1,$ $j$-я компонента есть вероятность того, что $l$ приборов заняты 
		обслуживанием заявок 1 типа и $MMAP$ находится в состоянии $j-1$
		
		$$
		{\bf q}_l=\sum\limits_{i=1}^N {\bf p}_i
		\begin{pmatrix}
		O_{l\bar {W}\times\bar {W}}\cr
		I_{\bar{W}}\cr
		O_{(i-l)\bar {W}\times \bar {W}}\cr
		\end{pmatrix},\,l=\overline{0,N}.
		$$
		
		\item[$\bullet$] Распределение числа  приборов, занятых обслуживанием заявок 1 типа
		
		$$
		q_l={\bf q}_l{\bf e}, \,l=\overline{0,N}.
		$$
		
		\item[$\bullet$] Среднее число  приборов, занятых обслуживанием заявок 1 типа
		
		$$
		N_{busy}^{(1)}=\sum\limits_{l=1}^N l q_l.
		$$
		
		\item[$\bullet$] Вектор ${\bf r}_l$ порядка $W+1,$ $j$-я компонента есть вероятность того, что $l$ приборов заняты 
		обслуживанием заявок 2 типа и $MMAP$ находится в состоянии $j-1$
		
		$$
		{\bf r}_l=\sum\limits_{i=1}^N {\bf p}_i
		\begin{pmatrix}
		O_{(i-l)\bar {W}\times\bar {W}}\cr
		I_{\bar{W}}\cr
		O_{l\bar {W}\times \bar {W}}\cr
		\end{pmatrix},\,l=\overline{0,N}.
		$$
		
		\item[$\bullet$] Распределение числа  приборов, занятых обслуживанием заявок 2 типа
		
		$$
		r_l={\bf r}_l{\bf e}, \,l=\overline{0,N}.
		$$
		
		\item[$\bullet$] Среднее число  приборов, занятых обслуживанием заявок 2 типа
		
		$$
		N_{busy}^{(2)}=\sum\limits_{l=1}^N l r_l.
		$$
		
		
		
		{\bf Замечание!} Здесь можно устроить проверку. Должно быть
		
		$$
		N_{busy}=N_{busy}^{(1)}+N_{busy}^{(2)}.
		$$
		
		
	\end{itemize}
	
	
	
	\section{ Вероятность потерь}
	
	Важнейшей из  характеристик системы  является  вероятность потерь в
	системе запросов разных типов. Согласно эргодической теореме для цепей Маркова, см.,
	например \cite{skor}, вероятность потери запроса   может быть вычислена как отношение разности интенсивности
	входного потока в систему  и выходящего  потока из системы  к
	интенсивности  входящего потока. Таким образом, справедлива
	следующая теорема.
	
	
	
	{\bf Теорема ~1.} {\itshape Вероятность потери запроса $k$-го типа  в системе
		рассчитывается как
		$$
		P_{loss,k}=\frac{\lambda_k-\varphi_k}{\lambda_k}
		\eqno(6)
		$$
		где $\lambda_k$ --  интенсивность поступления запросов $k$-го типа  в  $MMAP,$ $\varphi_k$ --  интенсивность выходящего 
		потока  запросов $k$-го типа  из системы.
		
		Значение  $\lambda_k $ вычисляется как
		$$\lambda_k={\bs \theta} D_k {\bf e},$$
		где вектор ${\bs \theta}$ единственное решение системы
		$${\bs \theta} (D_0+D_1+D_2)={\bf 0},\,\,\,{\bs \theta}{\bf e}=1,$$
		
		Значение  $\varphi_k$ вычисляется как
		$$
		\varphi_k=\mu^{(k)} N_{busy}^{(k)}, k=1,2.
		$$
	}
	
	
	
	
	{\bf Следствие~1.} {\itshape Вероятность потери произвольного запроса    рассчитывается как
		$$
		P_{loss}=P_{loss,1}+P_{loss,2}.
		$$
	}
	
	{\bf Замечание 2.} Вероятности потери запроса $k$-го типа и произвольного запроса можно также рассчитать, рассматривая 
	ситуацию в момент поступления  запросов в систему. Запрос  получит отказ, если все приборы в этот момент заняты. Тогда 
	вероятности отказа могут  быть вычислены по формуле
	
	$$
	P_{loss, k}=\frac{1}{\lambda_k}{\bf p}_N^{-}D_k{\bf e}, k=1,2,
	$$
	
	$$
	P_{loss}=\frac{1}{\lambda}{\bf p}_N^{-}(D_1+D_2){\bf e}.
	$$
	
	
	\begin{thebibliography}{20}
		
		\bibitem{he}
		Q.M. He, Queues with marked calls. Advances in Applied
		Probability, 28 (1996)  567-587.
		
		\bibitem{Graham81}
		Graham A.  Kronecker Products and Matrix Calculus with
		Applications /  A. Graham // Cichester:
		Ellis  Horwood, 1981.
		
		
		\bibitem{KlimKimOrlDud}	
		Klimenok V.I., Kim C.S., Orlovsky D.S., Dudin A.N. Lack of invariant property of  Erlang $BMAP/PH/N/0$ model // Queueing 
		Systems. 2005. V.49. P. 187-213.
		
		\bibitem{skor}
		Skorokhod A. Probability theory and random processes / Skorokhod A.
		// Kiev: High School, 1980.
		
		
		
	\end{thebibliography}
	
	
\end{document}